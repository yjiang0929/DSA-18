\documentclass{article}
\usepackage[utf8]{inputenc}

\title{\large{\textsc{In-Class 2: Linked Lists, Stacks and Queues}}}
\date{}

\usepackage{natbib}
\usepackage{graphicx}
\usepackage{amsmath}
\usepackage{amsfonts}
\usepackage{mathtools}
\usepackage[a4paper, portrait, margin=0.8in]{geometry}

\usepackage{listings}


\newcommand\perm[2][n]{\prescript{#1\mkern-2.5mu}{}P_{#2}}
\newcommand\comb[2][n]{\prescript{#1\mkern-0.5mu}{}C_{#2}}
\newcommand*{\field}[1]{\mathbb{#1}}

\DeclarePairedDelimiter\ceil{\lceil}{\rceil}
\DeclarePairedDelimiter\floor{\lfloor}{\rfloor}

\newcommand{\Mod}[1]{\ (\text{mod}\ #1)}

\begin{document}

\maketitle


\begin{lstlisting}[language=Java]
Node {
    int data;
    Node next;
}

\end{lstlisting}

\noindent For each of the following problems, assume you are only given a \texttt{Node head} (or two), which is the head of a linked list. Focus on first getting a working solution, and then optimizing to a O(N) time O(1) space.

\begin{enumerate}

\setcounter{enumi}{0}

\item Given two linked lists, check if they are equal.
    
\item Reverse a linked list. Return the new head node.

\item Check if a linked list is cyclic (that is, there is no node who's \texttt{next} pointer is \texttt{null}, because a node points to a previous node).

\item Check if a singly linked list is a palindrome. You can modify the linked list.
 
\end{enumerate}

\subsection*{}

Assumptions:

\begin{itemize}
  \item All stacks are listed from top to bottom.
  \item All problems can be solved using O(N) time and O(N) space unless noted otherwise.
  \item Stacks support: \texttt{push}, \texttt{pop}, \texttt{peek}, and \texttt{isEmpty}. They have no other methods.
\end{itemize}
    
\subsection*{}

\begin{enumerate}

\setcounter{enumi}{4}
    
\item Given an integer n, determine if it is palindromic (12,321 is palindromic, 23 is not) (Hint: Think about how you get the last digit of a number. i.e. 435 $\rightarrow$ 5).

\item Given a string of opening and closing parentheses, determine if it is a valid expression. That is, every opening paren has a closing partner eg \texttt{"()(())"} is valid, but \texttt{"()())(()"} is not.

\item How would you implement a queue using two stacks? You don’t have to actually implement it, but either draw it or write an explanation on the board. Explain its operations’ runtimes (enqueue and dequeue). Use amortized analysis to discuss your runtimes. Find an instructor and discuss your solution.
\end{enumerate}


\clearpage

\begin{lstlisting}[language=Python]

def equal(head1, head2):

    while (head1 is not None) and (head2 is not None):
        if head1.data != head2.data:
            return False
        head1 = head1.next
        head2 = head2.next

    return (head1 is None and head2 is None)

def reverseList(head):
    prev = None
    while head:
        curr = head
        head = head.next
        curr.next = prev
        prev = curr
    return prev

def hasCycle(head):
    slow = head
    fast = head
    while fast!=None and fast.next!=None:
        slow = slow.next
        fast = fast.next.next
        if slow is fast:
            return True
    return False

def find_nth(head, n):
    while n > 0 and head is not None:
        head = head.next
        n -= 1
    return head

def palindrome(head):

    l = length(head)

    if l <= 1:
        return True

    left_of_middle = find_nth(head, l/2 - 1)
    # if linked list has odd number of elements, drop the middle element
    if (l % 2) == 1:
        left_of_middle.next = left_of_middle.next.next

    # reverse the back half of the linked list
    left_of_middle.next = reverseList(left_of_middle.next)

    # seperate the front and back half of the LL into seperate Lists
    middle = left_of_middle.next
    left_of_middle.next = None

    # check equality
    return equal(head, middle)

def num_palindrome(i):
    j = i
    stack = Stack()
    while j != 0:
        stack.push(j%10)
        j/=10
    while not stack.isEmpty():
        if stack.pop() != i%10:
            return False
        i/=10
    return True

def valid_expr(s):
    stack = Stack()
    for c in s:
        if not stack.isEmpty() and c == ')' and stack.peek() == '(':
            stack.pop()
        else:
            stack.push(c)
    return stack.isEmpty()

\end{lstlisting}

\end{document}
