
\documentclass{article}
\usepackage[utf8]{inputenc}

\title{\large{\textsc{In-Class 7: Sorting in Problem Solving}}}
\date{}

\usepackage{natbib}
\usepackage{graphicx}
\usepackage{amsmath}
\usepackage{amsfonts}
\usepackage{mathtools}
\usepackage[a4paper, portrait, margin=0.8in]{geometry}

\usepackage{listings}

\newcommand\perm[2][n]{\prescript{#1\mkern-2.5mu}{}P_{#2}}
\newcommand\comb[2][n]{\prescript{#1\mkern-0.5mu}{}C_{#2}}
\newcommand*{\field}[1]{\mathbb{#1}}

\DeclarePairedDelimiter\ceil{\lceil}{\rceil}
\DeclarePairedDelimiter\floor{\lfloor}{\rfloor}

\newcommand{\Mod}[1]{\ (\text{mod}\ #1)}

\begin{document}

\maketitle

\begin{enumerate}


%%%%% PROBLEM 1 %%%%%
\item Given a sorted integer array without duplicates, return the number of ranges as follows:

\begin{itemize}
    \item Given the array \texttt{[0, 1, 2, 4, 5, 7]}, return 3. The ranges are: \texttt{[0$\rightarrow$2, 4$\rightarrow$5, 7]}.
\end{itemize}

%%%%% PROBLEM 2 %%%%%
\item Design a new data structure that contains ints that will allow the following operations in amortized $O(1)$ time.
\begin{itemize}
    \item \texttt{void insert(int value)}: Inserts the given item into the data structure
    \item \texttt{int remove(int value)}: Removes the given item from the data structure, if it exists
    \item \texttt{int getRandom()}: Return an item at random (each element should have equal probability of being chosen)
\end{itemize}

%%%%% PROBLEM 3 %%%%%
\item Given two integer arrays \texttt{A} and \texttt{B} sorted in ascending order, and a value \textbf{k}, a pair \textbf{(u,v)} is defined as consisting one element from \texttt{A} and one from \texttt{B}. Find \textbf{k} pairs with the smallest sum.

e.g. \texttt{A} = $[1,7,11]$, \texttt{B} = $[2,4,6]$, \textbf{k} = 3, return $[1,2], [1,4], [1,6]$.

e.g \texttt{A} = $[1,1,2]$, \texttt{B} = $[1,2,3]$, \textbf{k} = 2, return $[1,1], [1,1]$

e.g. \texttt{A} = $[1,2]$, \texttt{B} = $[3]$, \textbf{k} = 3, return all possible pairs $[1,3], [2,3]$

%%%%% PROBLEM 4 %%%%%
\item Given an UNSORTED integer array without duplicates, return the number of ranges as follows:

\begin{itemize}
    \item Given the array \texttt{[1, 4, 0, 5, 7, 2]}, return 3. The ranges are: \texttt{[0$\rightarrow$2, 4$\rightarrow$5, 7]}.
\end{itemize}
Can you do this in O(N) time?
\end{enumerate}

\clearpage


%%%%% SOLUTIONS %%%%%%


\begin{center}
    \textbf{Solutions}
\end{center}

\begin{enumerate}



%%%%% SOLUTION 1
\item Solution for Report Ranges

\begin{lstlisting}[language=Java]
public int ranges(int[] nums) {
    int ans = 0;
    for (int i = 0; i < nums.length; i++) {
        int a = nums[i];
        while (i + 1 < nums.length && (nums[i + 1] - nums[i]) == 1)
            i++;
        ans++;
    }
    return ans;
}
\end{lstlisting}

%%%%% SOLUTION 2
\item Solution for getRandom data structure

Use an array to store all the elements and a HashMap which maps from the elements to their index in the array. That way, you can get a random element from the array. When you insert an element, insert at the end and mark its index in the HashMap. When removing, swap the element you are removing with the last one so that the last element is retained. You can then pop off the last element. Make appropriate adjustments in the hash map.

\begin{lstlisting}[language=Python]
class RandomizedSet(object):

    def __init__(self):
        self.l=[]
        self.d={}

    def insert(self, val):
        if val in self.d:
            return
        self.d[val]=len(self.l)
        self.l.append(val)

    def remove(self, val):
        if val not in self.d:
            return
        index = self.d[val]
        if index == len(self.l)-1:
            self.l.pop()
        else:
            last_elem = self.l.pop()
            self.l[index] = last_elem
            self.d[last_elem] = index

        del self.d[val]

    def getRandom(self):
        if not self.l:
            return None
        return random.choice(self.l)

\end{lstlisting}

%%%%% SOLUTION 3
\item Solution for k pairs of smallest sum
\begin{lstlisting}[language=python]
# Pair definition is Pair(indexA, indexB)
# Runtime is O(klogk)
SMALLEST_K_PAIRS(A, B, k):
    answers = []
    pq = minPQ()
    # use sum as key. add pairs as values to PQ
    # use top row of possible sums as input to PQ. Make sure pq never grows
    # larger than size k. Once it is larger than k, dequeue the min and place
    # it in answers.
    for i in 0:len(B):
        pq.enqueue(A[0] + B[i], new Pair(0, i))
        if pq.size() > k:
            p = pq.dequeueMin()
            answers.add(p)
            if p.indexA < A.length - 1:
		pq.enqueue(A[p.indexA + 1] + B[p.indexB], new Pair(p.indexA + 1, p.indexB))
        if answers.length == k:
            return answers
    while answers.length < k:
        p = pq.dequeueMin()
        answers.add(p)
	if p.indexA < A.length - 1:
            pq.enqueue(A[p.indexA + 1] + B[p.indexB], new Pair(p.indexA + 1, p.indexB))

    return answers
}
\end{lstlisting}

%%%%% SOLUTION 4
\item Solution for unsorted range
\begin{lstlisting}[language=Java]
static String CountRange(String s) {
count = 0;
HashSet<Integer> mySet = new HashSet<>();
for (int i: arr){
    mySet.add(i)
}
        for (int i : arr){
            if (!mySet.contains(i-1)){count++}
        }
        return count;
}
\end{lstlisting}

\end{enumerate}

\end{document}
